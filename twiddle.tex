\documentclass{article}
\title{Twiddle -- A DSL for the Functional Bit-hacker}
\author{Michael Buch}

\usepackage[inline]{enumitem} % inline numbered lists
\usepackage[left=2cm,right=2cm]{geometry}
\usepackage{verbatim} % for comments

\begin{document}
\maketitle
\frenchspacing

\begin{abstract}
It is useful (and fun!) to bit twiddle i.e. perform arithmetic and manipulate data at the granularity of individual bits. Traditionally there is a compromise one has to make between using a low-level unsafe language that allows bit-twiddling versus a safe high-level language in which the type system or language design prohibit operations at bit-level (without additional complexity). The \textit{Twiddle} domain-specific language (DSL) is an embedded language written in Scala that generates bit-twiddling style C code. The language offers several modes of operation, useful for quick prototyping, debugging and custom extensions:
\begin{enumerate*}[label=(\arabic*)]
	\item Tracing interpreter
	\item Scala evaluator
	\item Twiddle AST generator
	\item C code generator
\end{enumerate*}.
Thus our language is a tool for the curious, a tool for safe bit-hackers and a tool for someone looking to get a bit more performance out of his high-level language.
\end{abstract}

\section{Motivation}
\section{The Language}
\subsection{Core}
The core of the language is split into modular pieces of functionality implemented as traits. As is common practice with tagless final interpreters (see section \ref{subsec:tagless}) we parameterize each set of language features with an evaluator that describes how each feature within its context. The core set of features is divided into following traits:
\begin{enumerate}
	\item Arithmatic
	\item Strings
	\item Bools
	\item Lambda
	\item LispLike
	\item CLike
	\item CMathOps
	\item CStrOps
\end{enumerate}
\subsection{Evaluators}
\section{Codegen}
\subsection{Twiddle AST}
\subsection{Twiddle AST Interpreters}
\section{Design Choices}
\subsection{Language Embedding}
\subsection{Scala}
\subsection{Tagless Final Style}\label{subsec:tagless}
\section{Conclusion \& Future Work}

\bibliographystyle{ieeetran}
\bibliography{twiddle}
\end{document}